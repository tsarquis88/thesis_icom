\chapter{Conclusión}

Se ha llevado a cabo el diseño de las bases del remplazo más moderno del módulo IAGC del radar Vitro. Si bien este sistema no se ha llevado aún a producción, ni se han hecho pruebas con las señales reales del receptor del radar, se han obtenido pruebas positivas que indican que no habría mayores problemas.\par

El uso de técnicas modernas de procesamiento de señales ha sido fructífero, obteniendo resultados muy cercanos a los esperados y/o simulados. Se demostró que es posible sacar provecho del \emph{aliasing} obtenido por un muestreo a una frecuencia menor a la de \emph{Nyquist}.\par

La FPGA estuvo a la altura de los requerimientos del sistema, pudiendo simplificar, en gran nivel, la implementación de la técnica de monopulso. De hecho, se cree que la misma placa podría realizar, de manera paralela, otras operaciones requeridas por el radar o de alguno de sus módulos.\par

Como trabajo futuro, se buscará hacer las pruebas pertinentes y llevar el sistema a producción, reemplazando asi el módulo IAGC. De esta manera, no solo se modernizará una parte del procesamiento del radar si no que también se ganará mucho espacio físico.\par